%% Generated by Sphinx.
\def\sphinxdocclass{report}
\documentclass[a4paper,11pt,portuges]{sphinxmanual}
\ifdefined\pdfpxdimen
   \let\sphinxpxdimen\pdfpxdimen\else\newdimen\sphinxpxdimen
\fi \sphinxpxdimen=.75bp\relax
\ifdefined\pdfimageresolution
    \pdfimageresolution= \numexpr \dimexpr1in\relax/\sphinxpxdimen\relax
\fi
%% let collapsible pdf bookmarks panel have high depth per default
\PassOptionsToPackage{bookmarksdepth=5}{hyperref}

\PassOptionsToPackage{booktabs}{sphinx}
\PassOptionsToPackage{colorrows}{sphinx}

\PassOptionsToPackage{warn}{textcomp}
\usepackage[utf8]{inputenc}
\ifdefined\DeclareUnicodeCharacter
% support both utf8 and utf8x syntaxes
  \ifdefined\DeclareUnicodeCharacterAsOptional
    \def\sphinxDUC#1{\DeclareUnicodeCharacter{"#1}}
  \else
    \let\sphinxDUC\DeclareUnicodeCharacter
  \fi
  \sphinxDUC{00A0}{\nobreakspace}
  \sphinxDUC{2500}{\sphinxunichar{2500}}
  \sphinxDUC{2502}{\sphinxunichar{2502}}
  \sphinxDUC{2514}{\sphinxunichar{2514}}
  \sphinxDUC{251C}{\sphinxunichar{251C}}
  \sphinxDUC{2572}{\textbackslash}
\fi
\usepackage{cmap}
\usepackage[T1]{fontenc}
\usepackage{amsmath,amssymb,amstext}
\usepackage{babel}



\usepackage{tgtermes}
\usepackage{tgheros}
\renewcommand{\ttdefault}{txtt}



\usepackage[Sonny]{fncychap}
\ChNameVar{\Large\normalfont\sffamily}
\ChTitleVar{\Large\normalfont\sffamily}
\usepackage{sphinx}

\fvset{fontsize=auto}
\usepackage{geometry}


% Include hyperref last.
\usepackage{hyperref}
% Fix anchor placement for figures with captions.
\usepackage{hypcap}% it must be loaded after hyperref.
% Set up styles of URL: it should be placed after hyperref.
\urlstyle{same}

\addto\captionsportuges{\renewcommand{\contentsname}{Sumário:}}

\usepackage{sphinxmessages}
\setcounter{tocdepth}{1}



\title{Análise de Sinais UFU - Notas de aula}
\date{26 set., 2025}
\release{1.0}
\author{Cristofer Antoni Souza Costa}
\newcommand{\sphinxlogo}{\vbox{}}
\renewcommand{\releasename}{Versão}
\makeindex
\begin{document}

\ifdefined\shorthandoff
  \ifnum\catcode`\=\string=\active\shorthandoff{=}\fi
  \ifnum\catcode`\"=\active\shorthandoff{"}\fi
\fi

\pagestyle{empty}
\sphinxmaketitle
\pagestyle{plain}
\sphinxtableofcontents
\pagestyle{normal}
\phantomsection\label{\detokenize{index::doc}}


\sphinxAtStartPar
Add your content using \sphinxcode{\sphinxupquote{reStructuredText}} syntax. See the
\sphinxhref{https://www.sphinx-doc.org/en/master/usage/restructuredtext/index.html}{reStructuredText}
documentation for details.

\sphinxstepscope


\chapter{Análise de Sinais UFU}
\label{\detokenize{introducao:analise-de-sinais-ufu}}\label{\detokenize{introducao::doc}}
\sphinxAtStartPar
Repositório para armazenar códigos, resultados computacionais e recursos de revisão bibliográfica relacionados ao estudo de sinais e sistemas, com ênfase especial em rotodinâmica e análise de falhas, desenvolvido como parte do meu mestrado.

\sphinxstepscope


\chapter{Da Incerteza à Previsão}
\label{\detokenize{motivacao:da-incerteza-a-previsao}}\label{\detokenize{motivacao::doc}}

\section{Como Sistemas Dinâmicos Revelam Ordem em Sinais Aleatórios}
\label{\detokenize{motivacao:como-sistemas-dinamicos-revelam-ordem-em-sinais-aleatorios}}
\sphinxAtStartPar
\sphinxstylestrong{Resumo:} A análise de sinais é um campo fundamental da engenharia e da física, tradicionalmente dividindo os sinais em duas categorias: determinísticos e aleatórios. Enquanto os primeiros são completamente previsíveis por meio de funções matemáticas, os últimos apresentam uma natureza imprevisível que desafia a análise direta. Este artigo explora como a interação de sinais aleatórios com sistemas dinâmicos — entidades físicas com massa, amortecimento e rigidez — atua como um processo de filtragem, revelando padrões e permitindo uma análise preditiva através de métodos estatísticos.


\section{A Dicotomia Fundamental: Sinais Determinísticos e Aleatórios}
\label{\detokenize{motivacao:a-dicotomia-fundamental-sinais-deterministicos-e-aleatorios}}
\sphinxAtStartPar
No campo do processamento de sinais, a primeira etapa de qualquer análise é a classificação do sinal. Conforme destacado pelo professor Duarte, essa classificação se divide em dois grandes grupos:

\sphinxAtStartPar
\sphinxstylestrong{Sinais Determinísticos:} São aqueles cujo comportamento pode ser descrito com precisão por uma função matemática explícita. O valor do sinal em qualquer instante futuro é conhecido, desde que suas condições iniciais e a função que o rege sejam conhecidas. Um exemplo clássico, citado na explanação, é o sinal de vibração gerado pelo desbalanceamento de um rotor. A não coincidência entre o centro de massa e o centro geométrico de rotação gera uma força centrífuga que se manifesta como uma vibração senoidal ou cossenoidal, perfeitamente modelável {[}\hyperlink{cite.motivacao:id6}{1}{]}. A equação que descreve essa força de desbalanceamento é dada por:
\begin{equation*}
\begin{split}F_u = m_u \cdot e \cdot \omega^2 \sin(\omega t)\end{split}
\end{equation*}
\sphinxAtStartPar
onde \(m_u\) é a massa desbalanceada, \(e\) é a excentricidade, e \(\omega\) é a velocidade angular.

\sphinxAtStartPar
\sphinxstylestrong{Sinais Aleatórios (ou Estocásticos):} Em contraste, são sinais cujo valor futuro não pode ser previsto com exatidão a partir de seus valores passados. Eles não seguem um padrão repetitivo e devem ser caracterizados por suas propriedades estatísticas, como média (\(\mu\)), variância (\(\sigma^2\)) e densidade espectral de potência (\(S_{xx}(f)\)). Exemplos comuns incluem o ruído térmico em um circuito eletrônico, a turbulência de um fluido ou as ondas do mar que atuam sobre um navio {[}\hyperlink{cite.motivacao:id7}{2}{]}.


\section{O Paradoxo da Análise Aleatória}
\label{\detokenize{motivacao:o-paradoxo-da-analise-aleatoria}}
\sphinxAtStartPar
A natureza imprevisível dos sinais aleatórios levanta uma questão fundamental: se não podemos prever o que acontecerá no próximo instante, qual o propósito de analisá\sphinxhyphen{}los? A resposta, como elucidado pelo professor Duarte, não está no sinal em si, mas na sua interação com um sistema dinâmico.

\sphinxAtStartPar
Um sistema dinâmico é qualquer sistema físico que possui a capacidade de armazenar e dissipar energia. Seus componentes fundamentais são massa (que armazena energia cinética), rigidez (que armazena energia potencial) e amortecimento (que dissipa energia). O comportamento desses sistemas é governado por leis físicas, como a Segunda Lei de Newton, que afirma que a somatória das forças é igual à massa vezes a aceleração:
\begin{equation*}
\begin{split}\sum F = m \cdot a\end{split}
\end{equation*}

\section{O Sistema Dinâmico como um Filtro Natural}
\label{\detokenize{motivacao:o-sistema-dinamico-como-um-filtro-natural}}
\sphinxAtStartPar
Quando um sinal aleatório atua como uma excitação (entrada) para um sistema dinâmico, o sistema não responde de forma totalmente aleatória. Em vez disso, ele atua como um filtro, moldando a resposta (saída) de acordo com suas próprias características intrínsecas.

\sphinxAtStartPar
Este conceito de «filtragem» ocorre porque todo sistema dinâmico possui frequências naturais (ou modos de vibração), que são as frequências nas quais o sistema tende a oscilar com maior amplitude quando perturbado. Quando o sinal de entrada aleatório, que pode ser considerado como contendo um amplo espectro de frequências (ruído branco, por exemplo), atravessa o sistema, as componentes de frequência próximas às frequências naturais do sistema são amplificadas, enquanto outras são atenuadas {[}\hyperlink{cite.motivacao:id8}{3}{]}.

\sphinxAtStartPar
Matematicamente, em um contexto digital, a resposta do sistema em um instante \(k\), \(x[k]\), depende dos valores passados da resposta e da excitação. Uma forma discretizada da equação do movimento pode ser expressa como uma equação de diferenças:
\begin{equation*}
\begin{split}x[k] = f(x[k-1], x[k-2], ..., u[k], u[k-1], ...)\end{split}
\end{equation*}
\sphinxAtStartPar
onde \(x[k]\) é a resposta no instante \(k\) e \(u[k]\) é a excitação. Esta dependência do passado é a própria definição de um filtro digital. A operação que descreve a interação entre o sinal de entrada e a resposta do sistema no domínio do tempo é a \sphinxstylestrong{convolução}:
\begin{equation*}
\begin{split}y(t) = \int_{-\infty}^{\infty} h(\tau) x(t-\tau) d\tau\end{split}
\end{equation*}
\sphinxAtStartPar
onde \(h(t)\) é a resposta ao impulso do sistema.


\section{Da Aleatoriedade à Análise Estatística Estruturada}
\label{\detokenize{motivacao:da-aleatoriedade-a-analise-estatistica-estruturada}}
\sphinxAtStartPar
O resultado deste processo é que o sinal de saída, embora ainda possua um caráter aleatório, não é mais «totalmente bagunçado». Ele carrega a «assinatura» do sistema dinâmico pelo qual passou. As frequências dominantes no sinal de resposta correspondem às frequências naturais do sistema.

\sphinxAtStartPar
Portanto, ao analisar estatisticamente o sinal de resposta — por exemplo, calculando sua \sphinxstylestrong{Densidade Espectral de Potência} (PSD):
\begin{equation*}
\begin{split}S_{yy}(f) = |H(f)|^2 S_{xx}(f)\end{split}
\end{equation*}
\sphinxAtStartPar
onde \(S_{xx}(f)\) é a PSD do sinal de entrada e \(H(f)\) é a função de transferência do sistema — podemos identificar esses picos de energia e, a partir deles, inferir propriedades críticas do sistema físico, como sua rigidez, amortecimento e possíveis danos estruturais {[}\hyperlink{cite.motivacao:id9}{4}{]}. É assim que a análise de vibrações em uma ponte excitada pelo vento (sinal aleatório) pode revelar sua saúde estrutural.


\section{Conclusão}
\label{\detokenize{motivacao:conclusao}}
\sphinxAtStartPar
A análise de sinais aleatórios, longe de ser um exercício fútil, é uma ferramenta poderosa para entender o mundo físico. A chave para desvendar a informação contida em um sinal aparentemente caótico é reconhecer o papel fundamental dos sistemas dinâmicos. Ao atuarem como filtros naturais, esses sistemas impõem uma ordem à aleatoriedade, permitindo que, através de ferramentas estatísticas, possamos extrair informações valiosas sobre suas próprias características. A aparente imprevisibilidade da entrada se transforma em conhecimento sobre o sistema, tornando a análise de sinais aleatórios um pilar essencial para o diagnóstico de falhas, o monitoramento de estruturas e o projeto de sistemas robustos.


\section{Referências}
\label{\detokenize{motivacao:referencias}}
\begin{sphinxthebibliography}{1}
\bibitem[1]{motivacao:id6}
\sphinxAtStartPar
D. J. Inman. \sphinxstyleemphasis{Engineering vibration}. Pearson, Boston, fourth edition edition, 2014. ISBN 978\sphinxhyphen{}0\sphinxhyphen{}13\sphinxhyphen{}287169\sphinxhyphen{}3.
\bibitem[2]{motivacao:id7}
\sphinxAtStartPar
Julius S. Bendat and Allan G. Piersol. \sphinxstyleemphasis{Random data: analysis and measurement procedures}. Wiley series in probability and statistics. Wiley, Hoboken, N.J, 4th ed edition, 2010. ISBN 978\sphinxhyphen{}0\sphinxhyphen{}470\sphinxhyphen{}24877\sphinxhyphen{}5.
\bibitem[3]{motivacao:id8}
\sphinxAtStartPar
Alan V. Oppenheim and Roland W. Schafer. \sphinxstyleemphasis{Discrete\sphinxhyphen{}time signal processing}. Always learning. Pearson, Harlow, third edition, pearson new international edition edition, 2014. ISBN 978\sphinxhyphen{}1\sphinxhyphen{}292\sphinxhyphen{}02572\sphinxhyphen{}8 978\sphinxhyphen{}1\sphinxhyphen{}292\sphinxhyphen{}03815\sphinxhyphen{}5.
\bibitem[4]{motivacao:id9}
\sphinxAtStartPar
Robert Bond Randall. \sphinxstyleemphasis{Vibration\sphinxhyphen{}based condition monitoring: industrial, aerospace and automotive applications}. Wiley, Chichester, 2011. ISBN 978\sphinxhyphen{}0\sphinxhyphen{}470\sphinxhyphen{}97766\sphinxhyphen{}8.
\end{sphinxthebibliography}



\renewcommand{\indexname}{Índice}
\printindex
\end{document}